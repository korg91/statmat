
Lezione del 27/05, ultima modifica 03/06, Michele Nardin


\section{Teoria dei Test più Potenti ed Uniformemente più Potenti}
Useremo la notazione: MP = most powerful e UMP = uniformly most powerful.\\
\\
\noindent \textbf{Test più potenti per verifica d'ipotesi semplici}\\
\\
Per ora ci concentriamo sui test semplici, ossia i test in cui le ipotesi determinano completamente la distribuzione della statistica sotto la data ipotesi.

Ricordiamo che per la verifica d'ipotesi riguardo ad un parametro $\vartheta$ della forma

$$\bigg \{
\begin{array}{rl}
H_0: & \vartheta = \vartheta_0 \\
H_1: & \vartheta = \vartheta_1 \\
\end{array}
$$

un dato test fornisce una regione di rifiuto $C$, (nota: quando ci riferiamo ad un test implicitamente ci riferiamo alla sua regione di rifiuto!) le cui probabilità d'errore sono

$$\bigg \{
\begin{array}{rl}
\alpha =&  P(\text{rifiuto } H_0 | H_0) \\
\beta =& P(\text{non rifiuto } H_0 | H_1) \\
\end{array}
$$

Fissato $\alpha$, ricordiamo che $1 - \beta$ è detto potenza del test. Possiamo sempre pensare $\beta$ come funzione di $\vartheta_1$:
$$\beta(\vartheta_1) = P(\text{non rifiuto } H_0 | \vartheta = \vartheta_1)$$

\begin{definizione}[test $MP_\alpha$]
 Fissato $\alpha \in (0,1)$, un test che minimizza $\beta(\vartheta_1)$ è detto test più potente di livello $\alpha$ ($MP_\alpha$) \small{(per la verifica d'ipotesi semplice contro alternativa semplice)}. 

\end{definizione}

Il nostro obiettivo sarà ovviamente quello di trovare tale test. Vediamo come i concetti visti fin'ora (soprattutto verosimiglianza e sufficienza) siano legati a quanto cerchiamo tramite il seguente

\begin{lemma}[Neyman - Pearson]
Sia $(X_1,...,X_n)$ un campione casuale, con variabili aventi distribuzione data dalla funzione $f_{X_i}(x, \vartheta)$.
Sia inoltre

$$\bigg \{
\begin{array}{rl}
H_0: & \vartheta = \vartheta_0 \\
H_1: & \vartheta = \vartheta_1 \\
\end{array}
$$

il sistema di ipotesi (entrambe semplici) da verificare. Indicata con $L(\vartheta;\vec{x})$ la funzione di ML, 
il test $MP_\alpha$ per la verifica di $H_0$ ha regione critica (o di rifiuto) data da
$$C = \{\vec{x} \in X : \frac{L(\vartheta_1;\vec{x})}{L(\vartheta_0;\vec{x})}>A\}$$
con A costante non negativa (che nella pratica dipenderà da $\alpha$)
\end{lemma}
\textbf{Nota:} tale risultato è abbastanza intuitivo: se $A$ è sufficientemente grande, 
quando si vede che $L(\vartheta_1;\vec{x}) \geq A \cdot L(\vartheta_0;\vec{x})$ 
è assurdo pensare che $\vartheta_0$ sia una scelta plausibile, in quanto lo è molto di più $\vartheta_1$!

\begin{proof}

Sia $C$ la regione critica del test (di livello $\alpha$) dato dal lemma, e sia $C^*$ la regione critica di un qualsiasi altro test (sempre di livello $\alpha$).

Per provare il lemma dobbiamo mostrare quindi che $\beta^* \geq \beta$ 
(che sono ovviamente le probabilità di errore del 2o tipo delle due regioni in considerazione)
(indicheremo nel seguito con $\overline{C}, \overline{C^*}$ i complementari delle regioni $C$ e $C^*$).
Notiamo subito che
$$\overline{C} = \overline{C} \cap ({C^*} \cup \overline{C^*}) = (\overline{C} \cap {C^*}) \cup (\overline{C} \cap \overline{C^*})$$
e che 
$$\overline{C}^* = \overline{C}^* \cap ({C} \cup \overline{C}) = (\overline{C}^* \cap {C}) \cup (\overline{C}^* \cap \overline{C})$$

\begin{align*}
\beta^* - \beta 
&= P(\vec{x} \in \overline{C}^* | \vartheta = \vartheta_1) 
	- P(\vec{x} \in \overline{C} | \vartheta = \vartheta_1)
\\ &= \int_{\vec{x} \in \overline{C}^* } L(\vartheta_1;\vec{x}) d \vec{x} 
	- \int_{\vec{x} \in \overline{C} } L(\vartheta_1;\vec{x}) d \vec{x}
\\ &= \int_{\vec{x} \in (\overline{C}^* \cap {C}) \cup (\overline{C}^* \cap \overline{C}) } L(\vartheta_1;\vec{x}) d \vec{x} 
	- \int_{\vec{x} \in (\overline{C} \cap {C^*}) \cup (\overline{C} \cap \overline{C^*}) } L(\vartheta_1;\vec{x}) d \vec{x}
\\ &= \int_{\vec{x} \in \overline{C}^* \cap C } L(\vartheta_1;\vec{x}) d \vec{x} 
	- \int_{\vec{x} \in \overline{C} \cap C^* } L(\vartheta_1;\vec{x}) d \vec{x}
\\ & \geq A \left[  \int_{\vec{x} \in \overline{C}^* \cap C } L(\vartheta_0;\vec{x}) d \vec{x} 
	- \int_{\vec{x} \in \overline{C} \cap C^* } L(\vartheta_0;\vec{x}) d \vec{x} \right]
\\ & = A \left[ \int_{\vec{x} \in (\overline{C}^* \cap {C}) \cup ({C}^* \cap {C}) } L(\vartheta_0;\vec{x}) d \vec{x} 
	- \int_{\vec{x} \in (\overline{C} \cap {C^*}) \cup ({C} \cap {C^*}) } L(\vartheta_0;\vec{x}) d \vec{x} \right]
\\ & = A \left[  \int_{\vec{x} \in C} L(\vartheta_0;\vec{x}) d \vec{x} 
	- \int_{\vec{x} \in C^* } L(\vartheta_0;\vec{x}) d \vec{x} \right]
\\ & = A ( \alpha - \alpha) = 0
\end{align*}

Da cui $\beta^* \geq \beta $
\end{proof}

\noindent \textbf{Importante:} Da nessuna parte abbiamo supposto che $\vartheta$ sia uno scalare, e guardando la dimostrazione notiamo che tale ipotesi sarebbe inutile: quindi (come al solito d'altronde) $\vartheta$ può benissimo essere un vettore di parametri incogniti!\\
\\
\noindent \textbf{Test uniformemente più potenti per verifica d'ipotesi semplici contro alternative composte}\\
\\
Vediamo come estendere i risultati della sezione precedente quando le ipotesi alternative sono un pochino più complesse:

A noi interessano soprattutto le ipotesi unilaterali o bilaterali, ossia del tipo

$$\bigg \{
\begin{array}{rl}
H_0: & \vartheta = \vartheta_0 \\
H_1: & \vartheta > (<) \vartheta_0 \text{ (oppure } \vartheta = \vartheta_1,\ \vartheta_1 > \vartheta_0 (\vartheta_1 < \vartheta_0) )\\
\end{array}
$$

$$\bigg \{
\begin{array}{rl}
H_0: & \vartheta = \vartheta_0 \\
H_1: & \vartheta \neq \vartheta_0 \text{ (oppure } \vartheta = \vartheta_1,\ \vartheta_1 > \vartheta_0 \vee \vartheta_1 < \vartheta_0 )\\
\end{array}
$$

\begin{definizione}[$UMP_\alpha$]
Un test per la verifica d'ipotesi semplice ($H_0$) contro alternativa composta ($H_1$) viene detto uniformemente più potente di livello $\alpha$ ($UMP_\alpha$) se esso è $MP_\alpha$ per ogni possibile ipotesi semplice contenuta in $H_1$ (cioè minimizza $\beta(\vartheta_1)$ per ogni possibile fissato $\vartheta_1$ contemplato in $H_1$).
\end{definizione}

Ovviamente nessuno ci assicura che tale test esista: vediamo due esempi, entrambi basati su campione casuale da normale, uno di esistenza (in cui vediamo anche come applicare il lemma) e uno di non esistenza.\\
\\
\textbf{Esempio:}[\textit{esistenza}] Sia $(X_1,...,X_n)$ da $N(\mu,\sigma^2)$ (varianza nota).
Vogliamo trovare un test $UMP_\alpha$ per le seguenti ipotesi:

$$\bigg \{
\begin{array}{rl}
H_0: & \mu = \mu_0 \\
H_1: & \mu > \mu_0 \\
\end{array}
$$

Per farlo, considero le ipotesi

$$\bigg \{
\begin{array}{rl}
H_0: & \mu = \mu_0 \\
H_1: & \mu = \mu_1 \\
\end{array}
$$

per $\mu_1 > \mu_0$ fissato e vado a costruire il test $MP_\alpha$ di Neyman-Pearson associato.
Innanzitutto troviamo che
\begin{align*}
\frac{L(\mu_1;\vec{x})}{L(\mu_0;\vec{x})} 
&= exp\{ -\frac{1}{2\sigma^2} \sum_{i=1}^n (x_i - \mu_1)^2 - (x_i - \mu_0)^2 \}
\\&= exp\{ -\frac{1}{2\sigma^2} \sum_{i=1}^n 2x_i(\mu_0 - \mu_1) +\mu_1^2 - \mu_0^2 \}
\end{align*}

Ci prepariamo ad applicare il lemma (prendiamo il logaritmo per semplificare i conti): notiamo che

$$ -\frac{1}{2\sigma^2} \sum_{i=1}^n 2x_i(\mu_0 - \mu_1) +\mu_1^2 - \mu_0^2 \geq ln(A)$$
se e solo se
$$\frac{\sum_{i=1}^n x_i}{n} \geq \frac{\sigma^2 ln(A)}{n(\mu_1 - \mu_0)} + \frac{\mu_1 + \mu_0}{2} (=:B)$$

Per rendere operativa la regola di decisione (ossia rifiuto $H_0$ se $\overline{X_n} \geq B$)
occorre specificare il valore di B. 
Sotto $H_0$, $$\overline{X_n} \sim N(\mu_0, \frac{\sigma^2}{n})$$
da cui, fissato un livello di confidenza $\alpha$, devo risolvere la seguente equazione in B:

$$P(\overline{X_n} \geq B | \mu=\mu_0) = \alpha$$

equivalente a

$$P(\frac{\overline{X_n} - \mu_0} {\sigma/\sqrt{n}} \geq \frac{B - \mu_0} {\sigma/\sqrt{n}} | \mu=\mu_0) = \alpha$$

Ma sotto $H_0$, $\frac{\overline{X}_n - \mu_0} {\sigma/\sqrt{n}} \sim N(0,1)$, quindi  $\frac{B - \mu_0} {\sigma/\sqrt{n}} = z_\alpha$ (restituito dalle tavole).

Definitivamente, la regione critica sarà

$$C=\{ \vec{x} \in X : \overline{x_n} \geq \mu_0 + z_\alpha \frac{\sigma}{\sqrt{n}} \}$$ 

Abbiamo quindi ritrovato il test che abbiamo usato nel capitolo precedente, senza passare per il concetto di statistica pivot.

Rimane da mostrare che questo test è $UMP_\alpha$. Ma ciò è già stato fatto, infatti il ragionamento precedente vale quale che sia $\mu_1 (> \mu_0)$ fissato!\\
\\
\textbf{Esempio:}[\textit{non esistenza}] Sia $(X_1,...,X_n)$ da $N(\mu,\sigma^2)$ (varianza nota).
Vogliamo trovare un test $UMP_\alpha$ per le seguenti ipotesi:

$$\bigg \{
\begin{array}{rl}
H_0: & \mu = \mu_0 \\
H_1: & \mu \neq \mu_0 \\
\end{array}
$$

Seguendo la falsariga di quanto visto nell'esempio precedente, possiamo ovviamente dividere l'ipotesi $H_1$ nei due seguenti sottocasi:

$$H_1: \mu = \mu_1,\ \mu_1 > \mu_0 \vee \mu_1 < \mu_0 $$

Per il primo caso ($ \forall \mu_1 > \mu_0$), come visto sopra, abbiamo la regione critica
$$C=\{ \vec{x} \in X : \overline{x_n} \geq \mu_0 + z_\alpha \frac{\sigma}{\sqrt{n}} \}$$ 
mentre per il secondo caso ($\forall \mu_1 < \mu_0$), in maniera del tutto analoga, avremo
$$C=\{ \vec{x} \in X : \overline{x_n} \leq \mu_0 - z_\alpha \frac{\sigma}{\sqrt{n}} \}$$  
Notiamo quindi che le due regioni critiche non coincidono su ogni ipotesi semplice contenuta in $H_1$, e quindi (per definizione) non esiste il test $UMP_\alpha$ per verifica d'ipotesi bilaterali.\\
\\
Ci poniamo ora il problema di vedere quando esistono i test $UMP_\alpha$.\\
\\
\noindent \textbf{Osservazione:}[\textit{Legame tra statistiche sufficienti e test di Neyman-Pearson}]
Supponiamo di avere un campione casuale da una distribuzione con funzione di densità $f(x;\vartheta)$.
Supponiamo inoltre che $T_n$ sia una statistica sufficiente per $\vartheta$.
In accordo con il teorema di fattorizzazione, risulta che possiamo scrivere la funzione di verosimiglianza come

$$L(\vartheta;\vec{x}) = h(\vec{x}) g(t_n(\vec{x}),\vartheta)$$

ove $h$ non dipende da $\vartheta$.
Quindi, il \textit{rapporto di verosimiglianze (RV)} può essere scritto come

$$RV(\vartheta_1,\vartheta_0; \vec{x}) = \frac{L(\vartheta_1;\vec{x})}{L(\vartheta_0;\vec{x})} = \frac{g(t_n(\vec{x}), \vartheta_1)}{g(t_n(\vec{x}), \vartheta_0)}$$

Segue che, avendo a disposizione una statistica sufficiente per $\vartheta$, tale rapporto dipende da $\vec{x}$ solo attraverso $t_n$.

Introduciamo il concetto di \textit{monotonia del rapporto di verosimiglianza}.

\begin{definizione}
Nelle ipotesi dell'osservazione qua sopra, nel caso in cui $\vartheta_1 < \vartheta_0$ ($\vartheta_1 > \vartheta_0$) diciamo che $RV(\vartheta_1,\vartheta_0; \vec{x})$ è monotono se esso è una funzione crescente (decrescente) di $t_n$. 
\end{definizione}

Intuitivamente, se il rapporto tra le verosimiglianze è monotono, è facile immaginare che il test $UMP_\alpha$ per ipotesi unilaterali esiste sempre!
\\
Ciò è sicuramente vero se ci restringiamo a famiglie esponenziali:
consideriamo la famiglia esponenziale nella sua forma più semplice, ossia con funzione di densità della forma
$$f_X(x;\theta) = \exp \{ A(\theta) B (x) +  C(x) +  D(\theta) \}$$

\begin{lemma}
Un campione casuale da una famiglia esponenziale con $A(\vartheta)$ monotona ammette $RV$ monotono.
\end{lemma}

\begin{proof}
Infatti, 
\begin{align*}
RV(\vartheta_1,\vartheta_0; \vec{x}) 
&= \frac{\exp \{ A(\vartheta_1) \sum_i B (x_i) + \sum_i C(x_i) +  n D(\vartheta_1) \}}{\exp \{  A(\vartheta_0) \sum_i B (x_i) +  \sum_i C(x_i) +  n D(\vartheta_0) \}}
\\&= \exp \{ [A(\vartheta_1) - A(\vartheta_0)] \cdot \sum_i B(x_i) + n[ D(\theta_1) -  D(\theta_0)] \}
\end{align*}

Essendo $A$ monotona, si nota subito che se $\vartheta_1 > \vartheta_0$ allora $A(\vartheta_1) - A(\vartheta_0)\geq 0$ e quindi $RV$ è una funzione crescente rispetto a $\sum_i B (x_i)$
\end{proof}

Da quanto visto sopra, segue immediatamente che, testando le ipotesi 

$$\bigg \{
\begin{array}{rl}
H_0: & \vartheta = \vartheta_0 \\
H_1: & \vartheta < \vartheta_0 \\
\end{array}
$$

preso un qualunque $\vartheta_1 < \vartheta_0$, la condizione 
$$RV(\vartheta_1,\vartheta_0;\vec{x}) \leq k$$

è equivalente a 

$$\sum_{i=1}^n B(x_i) \leq c$$ per una costante $c$ ricavabile facilmente dall'equazione sopra, e valido per ogni $\vartheta_1 < \vartheta_0$. Questo fornisce il tesi $UMP_\alpha$
per la famiglia esponenziale (in questo caso vista nella forma semplice) per test unilaterali. Ovviamente, nel caso in cui $H_1: \vartheta > \vartheta_0$, la regione critica sarà fornita da $\sum_{i=1}^n B(x_i) \geq c$.\\

\subsection{Rapporto di massime verosimiglianze}

Introduciamo un test valido in generale, con il defetto di esser senza garanzie (se non asintotiche) di ottimalità ma con il pregio di non aver bisogno di particolari forme d'ipotesi. 

\begin{definizione}
Sia $(X_1,...,X_n)$ un campione casuale da una distribuzione avente funzione di ripartizione $F(x,\vartheta)$, con $\vartheta \in \Theta$.
Supponiamo di avere un sistema di ipotesi
$$\bigg \{
\begin{array}{rl}
H_0: & \vartheta \in \Theta_0 \\
H_1: & \vartheta \in \Theta \setminus \Theta_0 \\
\end{array}
$$

Definiamo la funzione Rapporto di Verosimiglianza generalizzato (RV generalizzato) come la funzione $\lambda: \mathcal{X} \in \mathbb{R}^n \to [0,1]$
definita da
$$\lambda(\vec{x}) = \frac{\max_{\vartheta \in \Theta_0} L(\vartheta,\vec{x})}{\max_{\vartheta \in \Theta} L(\vartheta,\vec{x})}$$
\end{definizione}

$\lambda$ sta in $[0,1]$ poiché, essendo $L$ una funzione di densità, $\lambda \geq 0$, ed invece $\lambda \leq 1$ essendo $\Theta_0 \subset \Theta$.\\
\\
Fissato un $A \in (0,1)$, una regione di rifiuto del test è data da 
$$C := \{ \vec{x} \in \mathcal{X}: \lambda(\vec{x})<A \}$$
Ovviamente $A$ sarà scelto in modo che 
$$P(\lambda(\vec{x})<A | H_0) = \alpha$$
con $\alpha$ probabilità (fissata a priori) di commettere un errore del primo tipo.\\
\\
La potenza di questo metodo è data dal seguente

\begin{teo}
Sotto $H_0$ la statistica (test di Wald) $W=-2ln[\lambda(\vec{x})]$, converge in distribuzione (all'aumentare di n) a $\chi^2_\nu$, ove $\nu$ è la differenza tra i parametri da stimare sotto $H_1$ rispetto ad $H_0$.
\end{teo}

Quindi, per campioni numerosi, è possibile ricavare agevolmente il parametro $A$ dalle tavole della $\chi^2$.\\
\\
\noindent \textbf{Esempio:} Sia $(X_1,...,X_n)$ un campione casuale da Poisson con parametro $\theta$, e quindi $$f(x,\theta) = P(X_i = x) = \frac{e^{-\theta} \theta^x}{x!} \mathbb{I}_{(0,1,2,...)}(x)$$ e $\theta>0$.
$$\bigg \{
\begin{array}{rl}
H_0: & \theta = \theta_0  \\
H_1: & \theta \neq \theta_0 \ \ [\theta \in \mathbb{R}^+ \setminus \{\theta_0\}]\\
\end{array}
$$
Calcoliamo inoltre la funzione di verosimiglianza:
$$L(\theta,\vec{x})=\frac{e^{-n\theta} \theta^{\sum x_i}}{\prod_{i=1}^n x_i!}$$
troviamo che, posto $\hat{\theta} = \overline{X}_n$ lo stimatore di massima verosimiglianza
$$\lambda(\vec{x}) = \frac{\max_{\theta \in \Theta_0} L(\theta,\vec{x})}{\max_{\theta \in \Theta} L(\theta,\vec{x})} = \frac{L(\theta_0,\vec{x})}{L(\hat{\theta},\vec{x})}$$

(notare che $\max_{\theta \in \Theta} L(\theta,\vec{x})=L(\hat{\theta},\vec{x}) $ proprio per definizione di stimatore di massima verosimiglianza!)

$$\lambda(\vec{x}) = \dfrac{ e^{-n\theta_0} \theta_0^{\sum x_i}}{e^{-n\hat{\theta}} \hat{\theta}^{\sum x_i}} = 
\left( \frac{\theta_0}{\hat{\theta}} \right)^{\sum x_i} e^{n(\hat{\theta} - \theta_0)}$$

Usiamo la statistica test di Wald:
$W = -2 ln (\lambda (\vec{x})) =  -2\left[\sum x_i ln \left( \frac{\theta_0}{\hat{\theta}} \right) + n(\hat{\theta} - \theta_0) \right]$
la quale, sotto $H_0$, $W \stackrel{D}{\longrightarrow} \chi^2$.
La regione critica che cerchiamo è data da
$$C := \{ \vec{x} \in \mathcal{X}: \lambda(\vec{x})<A \}$$
Vale $\lambda(\vec{x})<A \Leftrightarrow -2ln[ \lambda(\vec{x})]> -2ln(A)$.
Fissato $\alpha$, se $n$ è abbastanza grande possiamo approssimare
 $$P(\lambda(\vec{x})<A \Bigm\vert H_0) = P(-2ln[ \lambda(\vec{x})]> -ln(A^2) \Bigm\vert H_0) \cong P(\chi^2_1> -ln(A^2)\Bigm\vert H_0) = \alpha$$ sse $$-ln(A^2) = \chi_{1;\alpha/2}^2$$ da cui 

\begin{align*}
C  &= \{ \vec{x} \in \mathcal{X}: -2\left[\sum x_i ln \left( \frac{\theta_0}{\hat{\theta}} \right) + n(\hat{\theta} - \theta_0) \right] >  \chi_{1;\alpha/2}^2 \}
\end{align*}